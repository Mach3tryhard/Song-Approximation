\documentclass[english]{article}
\usepackage{graphicx}
\usepackage{subfigure}
\usepackage[T1]{fontenc}
\usepackage[utf8]{inputenc}
\usepackage{babel}
\usepackage{amsmath}
\usepackage{amsthm}
\usepackage{amssymb}
\usepackage{amsfonts}
\usepackage{enumerate}
\usepackage{fancyhdr}
\usepackage{float}
\usepackage{url}
\usepackage{hyperref}
\usepackage{amsmath}
\usepackage{natbib}
\newcommand{\LaTeXs}{\LaTeX{\/} }
\newcommand{\matlabs}{{\sc Matlab{\/} }}
\newcommand{\matlab}{{\sc Matlab}}
\begin{document}

\title{Aproximarea unei melodii folosind mai multe metode de calcul numeric}

\author{
  Popa Mircea Alex \\
  Sîrghe Matei Ștefan \\
  Ungureanu Robert
}

\maketitle

\section{Modelul matematic}\label{Modelul}
\begin{itemize}
\item Modelul ales de noi începe de la orice fișier audio de tip .wav.\\
\item Fișierul audio poate să fie interpretat ca o amplitudine în funcție de timp. Astfel, putem să ne imaginăm că avem o funcție de tipul $f(t)$, unde $t$ este timpul și $f(t)$ este amplitudinea sunetului la acel moment.\\
\item Ca să interpretăm funcția cu acuratețe mai mare, vom aplica Short-Time Fourier Transform, care ne va permite să vedem frecvențele care compun sunetul.\\
\item Astfel o sa ajungem cu o functie $F(\omega,t)$ cu care o sa putem sa modelăm o spectrogramă și pe care o să o putem interpreta ca o matrice.\\
\item Vom aplica SVD (Singular Value Decomposition) pe matricea $F(\omega,t)$, folosind factorizarea QR.\\
\item Am aplicat SVD, vom obține matricea $F(\omega,t) \approx U \Sigma V^T$, unde $U$ și $V$ sunt matricele ortogonale și $\Sigma$ este o matrice diagonală cu valorile pozitive.\\
\item Inmulțim cele 3 matrici, obținem o aproximare la matricea originală $G(\omega,t)$. \\ 
\item Cu matricea $G(\omega,t)$ vom calcula Short-Time Fourier Transform inversă, care ne va permite să obținem o nouă funcție $g(t)$, aproximarea funcției originale $f(t)$.\\
\item Cu aproximarea $g(t)$ vom putea să obținem un nou fișier audio .wav, care este o aproximare a fișierului original.\\
\end{itemize}

\begin{equation}
\begin{cases}
F(\tau,\omega) = \Sigma_{t=0}^{N-1} f(t) w(t-\tau) e^{-i\omega t} \\[6pt]
g(t) = \frac{1}{w(t-\tau)} \frac{1}{2\pi} \Sigma_{\omega=0}^{N-1} G(\tau,\omega) e^{+i\omega t} \\[6pt]
F(i,j) = A_{i,j} , A \in \mathbb{M}_{N,M} (\mathbb{C}) \\[6pt]
G(i,j) = B_{i,j} , B \in \mathbb{M}_{N,M} (\mathbb{C}) \\[6pt]
A \approx U\Sigma V^T = B \\[6pt]
w(n)=\frac{1}{2}[1+cos(\frac{2\pi n}{L})] , w:\mathbb{R} \rightarrow \mathbb{R} \\[6pt]
\end{cases}
\end{equation}

\begin{itemize}
	\item $w(n)$ este window function Hann
    \item $A$ și $B$ sunt matricele de amplitudine și frecvență
	\item $B$ se calculează utilizând metoda SVD și factorizarea Gram-Schmidt
\end{itemize}

\section{Discretizarea domeniului}\label{methods}

Domeniul este reprezentat de catre durata melodiei care este înmultită cu sampling rate-ul de 44100 Hz.

\begin{equation}
N = \text{l} \times h
\end{equation}

\begin{itemize}
	\item $l$ este durata melodiei în secunde
    \item $h$ este sampling rate-ul default de 44100 Hz
\end{itemize}

\section{Transformata Fourier}\label{Modelul}

Teoretic vorbind, Short-Time Fourier Transform este:
\begin{equation}
    F(\tau,\omega)=\int_{-\inf}^{+\inf}f(t)w(t-\tau)e^{-i\omega t}dt
\end{equation}
 dar funcția $f(t)$ este discretizată, deci vom folosi suma discretă ca să calculăm $F(\tau,\omega)$.

\begin{equation}
    F(\tau,\omega) = \Sigma_{t=0}^{N-1} f(t) w(t-\tau) e^{-i\omega t}
\end{equation}

Cu funcția $F(\tau,\omega)$ vom calcula spectograma originală pe care o vom nota cu $S_{o}$\\

\begin{equation}
	S_{o} = \begin{bmatrix}
		F(\tau_{1},\omega_{M})^{2} & F(\tau_{2},\omega_{M})^{2} & ... & F(\tau_{N},\omega_{M})^{2} \\
        ... & ... & ... & ... \\
		F(\tau_{1},\omega_{2})^{2} & F(\tau_{2},\omega_{2})^{2} & ... & F(\tau_{N},\omega_{2})^{2} \\
        F(\tau_{1},\omega_{1})^{2} & F(\tau_{2},\omega_{1})^{2} & ... & F(\tau_{N},\omega_{1})^{2} \\
	\end{bmatrix}
\end{equation}

\section{Sistemului Liniar}\label{Modelul}

In același timp, vom avea matricea:

\begin{equation}
	A = \begin{bmatrix}
		F(\tau_{1},\omega_{1}) & F(\tau_{1},\omega_{2}) & ... & F(\tau_{1},\omega_{M}) \\
        F(\tau_{2},\omega_{1}) & F(\tau_{2},\omega_{2}) & ... & F(\tau_{2},\omega_{M}) \\
        ... & ... & ... & ... \\
        F(\tau_{N},\omega_{1}) & F(\tau_{N},\omega_{2}) & ... & F(\tau_{N},\omega_{M}) \\
	\end{bmatrix}
\end{equation}

Teoretic vorbind, factorizarea SVD este:
\begin{equation}
    A = U \Sigma V^T
\end{equation}
 dar o să avem eroare cauzată de floating point, deci vom obține o matrice aproximativă $B$ care este diferită de cea de la care am pornit.

\begin{equation}
    B = U \Sigma V^T \approx A
\end{equation}

Putem să creăm o funcție $G(\tau,\omega)$ care primeste valorile din matricea $B$ și care este definită ca:
\begin{equation}
    G(\tau,\omega) = \begin{cases}
        B_{i,j} & \text{dacă } i \leq N \text{ și } j \leq M \\[6pt]
        0 & \text{altfel}
    \end{cases}
\end{equation}

Utilizând noua funcție, putem crea o nouă spectogramă care o să ne arate eroarea de aproximare pe care o avem:

\begin{equation}
	S_{e} = \begin{bmatrix}
		|F(\tau_{1},\omega_{M})^{2} - G(\tau_{1},\omega_{M})^{2}| & |F(\tau_{2},\omega_{M})^{2} - G(\tau_{2},\omega_{M})^{2}| & ... & |F(\tau_{N},\omega_{M})^{2} - G(\tau_{N},\omega_{M})^{2}| \\
        ... & ... & ... & ... \\
		|F(\tau_{1},\omega_{2})^{2} - G(\tau_{1},\omega_{2})^{2}| & |F(\tau_{2},\omega_{2})^{2} - G(\tau_{2},\omega_{2})^{2}| & ... & |F(\tau_{N},\omega_{2})^{2} - G(\tau_{N},\omega_{2})^{2}| \\
        |F(\tau_{1},\omega_{1})^{2} - G(\tau_{1},\omega_{1})^{2}| & |F(\tau_{2},\omega_{M})^{2} - G(\tau_{2},\omega_{M})^{2}| & ... & |F(\tau_{N},\omega_{1})^{2} - G(\tau_{N},\omega_{1})^{2}| \\
	\end{bmatrix}
\end{equation}

\section{Transformata Fourier Inversă}\label{Modelul}

Calculam aproximarea funcției originale $f(t)$ folosind Short-Time Fourier Transform inversă.

Teoretic vorbind, Short-Time Fourier Transform inversă este:
\begin{equation}
    g(t) = \frac{1}{w(t-\tau)} \frac{1}{2\pi} \int_{-\inf}^{+\inf} G(\tau,\omega) e^{+i\omega t} d\omega
\end{equation}

dar funcția $G(\tau,\omega)$ este discretizată, deci vom folosi suma discretă ca să calculăm $g(t)$.

\begin{equation}
    g(t) = \frac{1}{w(t-\tau)} \frac{1}{2\pi} \Sigma_{\omega=0}^{N-1} G(\tau,\omega) e^{+i\omega t} \\[6pt]
\end{equation}

Cu funcția $g(t)$ vom creea o nouă funcție care reprezintă eroarea la un anumit timp $f_{ea}$ și funcția $f_{cea}$ care reprezină eroarea cumulată a aproximării.
\begin{equation}
    f_{cea} = \Sigma_{t=0}^{N-1} |f(t) - g(t)| \\[4pt]
\end{equation}
\begin{equation}
    f_{ea} = |f(t) - g(t)|
\end{equation}

\end{document}
